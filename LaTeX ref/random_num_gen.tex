% !TeX TXS-program:bibliography = bibtex8 --huge --csfile "utf8cyrillic.csf" %
\UseRawInputEncoding
\documentclass[bachelor, och, referat, times]{SCWorks}
\usepackage[utf8]{inputenc}
\usepackage[T1]{fontenc}
\usepackage[english, russian]{babel}
\usepackage{minted}
\usepackage{amsmath}

\chair{математической кибернетики и компьютерных наук}
\course{1}
\group{111}
\napravlenie{02.03.02 "Фундаментальная информатика и информационные технологии"}
\title{Генерация случайных чисел}
\author{Карасева Арсения Михайловича}
\satitle{Программист}
\saname{В.\,С.\,Петров}
\date{2021}

\begin{document}
\renewcommand\theFancyVerbLine{\small\arabic{FancyVerbLine}}
\maketitle
\tableofcontents

\section{Какими бывают случайные числа?}
Стоит начать с того, какими бывают случайные числа. Существует два вида случайных чисел:
истинные случайные числа и псевдослучайные числа.

\subsection{Истинные случайные числа}
Истинные случайные числа можно получить из непредсказуемых физических явлений.
Работа устройств, генерирующих такие числа, основана на использовании надёжных источников энтропии, 
таких, как тепловой шум, дробовой шум, фотоэлектрический эффект, квантовые явления и т. д. 
Эти процессы в теории абсолютно непредсказуемы, на практике же получаемые из них 
случайные числа проверяются с помощью специальных статистических тестов.
Например, анализ уровня шумов звуковой карты компьютера.
Последние байты этой величины будут являться истинными случайными числами. \cite{trueRandGen}

\subsection{Псевдослучайные числа}
У псевдослучайных чисел есть некоторый математический алгоритм, по которому они генерируются.
За основу берётся какое-то стартовое число (его называют seed). Соответственно, Генератор псевдослучайных чисел
(ГПСЧ) --- алгоритм, генерирующий последовательность чисел, элементы которой почти независимы друг от друга 
и подчиняются заданному распределению. Каждое новое число в последовательности ГПСЧ генерируется исходя из 
предыдущего определенным способом. \cite{randNumGen}

Пример самого примитивного ГПСЧ:
\begin{minted}[mathescape, linenos, numbersep=5pt, frame=lines, framesep=2mm]{cpp}
int PRNG()
{
	static unsigned long int seed = 5643;
	seed = seed * 1103515245 + 12345;
	return (unsigned int)(seed / 65536) % 32768;
}
\end{minted}
Этот алгоритм не является хорошим, из--за своего распределения вероятностей — очень маленький шанс <<выбить>> маленькое число.
А что же такое распределение?
\section{Распределения}
\subsection{Ключевые термины}
\begin{itemize}
\item
Распределение вероятностей --- это закон, описывающий область значений случайной величины и соответствующие вероятности появления этих значений. Грубо говоря, распределения отвечают за вероятность, с которой определённые числа будут генерироваться.
\item
Плотность вероятности --- вещественная функция, характеризующая сравнительную вероятность реализации тех или иных значений случайной переменной. 
\item
Функция распределения в теории вероятностей --- функция, характеризующая распределение случайной величины.
\item
Математическое ожидание ($\mu$) --- среднее значение случайной величины. В случае непрерывной случайной величины подразумевается взвешивание по плотности распределения. \cite{Gnedenko2019}
\item
Среднеквадратическое отклонение ($\sigma$) --- наиболее распространённый показатель рассеивания значений случайной величины относительно её математического ожидания.
\end{itemize}
\subsection{Равномерное распределение}
Равномерное распределение в теории вероятностей --- распределение случайной вещественной величины, принимающей значения, принадлежащие некоторому промежутку конечной длины, характеризующееся тем, что плотность вероятности на этом промежутке почти всюду постоянна. То есть вероятности появлений всех чисел в диапазоне равны, соответственно, это самое подходящее распределение для ГПСЧ.

Математическое ожидание в этом распределении равно $\mu = (b-a) / 2$, то есть для $a = 2$ и $b = 6$ самое ожидаемое число будет 4. Функция rand() в C++ имеет равномерное распределение.

\subsection{Нормальное распределение (Гауссово)}
Нормальное распределение --- распределение вероятностей, которое в одномерном случае задаётся функцией плотности вероятности, совпадающей с функцией Гаусса:

$ f(x)={\frac {1}{\sigma {\sqrt {2\pi }}}}\;\exp {\biggl (}{-{\frac {(x-\mu )^{2}}{2\sigma ^{2}}}}{\biggr )} $

Таким образом, одномерное нормальное распределение является двухпараметрическим семейством распределений, которое принадлежит экспоненциальному классу распределений. \cite{Chistyakov1987} Стандартным нормальным распределением называется нормальное распределение с математическим ожиданием $\mu = 0$ и стандартным отклонением $\sigma = 1$. \cite{randNumGenDist}

Я реализовал генерацию случайных чисел на нормальном распределении, всё свелось к анализу графика функции нормального распределения для получения значения в некотором промежутке. Код реализации:
\begin{minted}[mathescape, linenos, numbersep=5pt, frame=lines, framesep=2mm]{cpp}
#include <iostream>
#include <vector>
#include <math.h>
#include <iomanip>
#include <map>
#define sqrt_2 1.414213562373
#define sqrt_pi_2 1.253314137316
#define eps 1e-5


const double values[] = { 0.0,        //reversedFunction(0.5)
                      0.125661,       //reversedFunction(0.55)
                      0.253347,       //reversedFunction(0.6)
                      0.385320,       //reversedFunction(0.65)
                      0.524401,       //reversedFunction(0.7)
                      0.674490,       //reversedFunction(0.75)
                      0.841621,       //reversedFunction(0.8)
                      1.036433,       //reversedFunction(0.85)
                      1.281552,       //reversedFunction(0.9)
                      1.554774,       //reversedFunction(0.94)
                      1.644854,       //reversedFunction(0.95)
                      1.880794,       //reversedFunction(0.97)
                      2.053749,       //reversedFunction(0.98)
                      2.326348 };     //reversedFunction(0.99)


double normalFunction(double x)
{
    if (x == 0.5) {
        return 0.0;
    }
    if (x < eps) {
        return -4.3;
    }
    if (x > 1 - eps) {
        return 4.3;
    }


    int sign = 1;
    if (x < 0.5) {
        x = 1 - x;
        sign = -1;
    }
    double xk1 = values[(int)(round(10 * x) - 5)], xk = xk1;
    double b = 1 - 2 * x;
    do 
    {
        xk = xk1;
        double t = xk / sqrt_2;
        xk1 = xk - sqrt_pi_2 * exp(t * t) * (erf(t) + b);
    } 
    while (fabs(xk - xk1) >= eps);


    return sign == 1 ? xk1 : -xk1;
}


double getNormRandVal()
{
    return normalFunction((double)rand() / ((double)RAND_MAX));
}


double finalResult(double m, double sigma)
{
    std::vector<double> vec;
    double randNum;
    randNum = m + sigma * getNormRandVal();


    return randNum;
}
\end{minted}
Получается, в данном случае мы <<переводим>> равномерное распределение в нормальное.

\subsection{Экспоненциальное распределение}
Экспоненциальное распределение моделирует время между двумя последовательными свершениями одного и того же события.
Для подсчёта вероятности нужен один параметр ($\lambda$), причём $\lambda > 0$. \cite{randNumGenDist} Тогда плотность вероятности этого распределения
имеет вид:

\begin{equation*}
f_{X}(x) = 
 \begin{cases}
   \lambda\,e^{{-\lambda x}}, x\ge 0, \\
   0, x<0.
 \end{cases}
\end{equation*}


Пример: пусть существует магазин, в который время от времени заходят покупатели. При определённых допущениях время между появлениями двух последовательных покупателей будет случайной величиной с экспоненциальным распределением. Среднее время ожидания нового покупателя (<<время затухания>>) = $ \frac{1}{\lambda} $ , а сама $\lambda$ --- среднее число покупателей за единицу времени.

Также математическим ожиданием для экспоненциального распределения является $ \frac{1}{\lambda} $. 
Соответственно, чем меньше лямбда, тем больше время затухания.

Реализация алгоритма генерации случайных чисел на экспоненциальном распределении:

\begin{minted}[mathescape, linenos, numbersep=5pt, frame=lines, framesep=2mm]{cpp}
double getRandomnumber()
{
	return (double)rand() / (double)RAND_MAX;
}


double exponentialDistribution(double ly)
{
	double x, u;
	u = getRandomnumber();
	x = log(1 - u) * (1 / (-ly)); // Случайное число, взятое
	return x;   // из экспоненциального распределения.
}
\end{minted}

\bibliography{thesis}
\bibliographystyle{gost780uv}

\end{document}
